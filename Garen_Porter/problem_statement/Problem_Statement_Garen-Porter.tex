\documentclass[10pt,draftclsnofoot,onecolumn,letterpaper]{article}
    \usepackage{ragged2e}
    \usepackage[svgnames,table]{xcolor}
    \usepackage[hidelinks]{hyperref}
    
    \usepackage[utf8]{inputenc}
    \usepackage[left=0.75in, right=0.75in, top=0.75in]{geometry}
    
    \usepackage[T1]{fontenc}
    \setlength{\parindent}{0pt}
    
    
    \begin{document}
    \begin{Center}
    {\fontsize{14pt}{16.8pt}\selectfont OPEnS Pyranometer: IoT Solar Radiation Sensor\par}
    \end{Center}\par
    
    \begin{Center}
    Garen Porter
    \end{Center}\par
    
    \begin{Center}
    CS 461: Senior Software Engineering Project
    \end{Center}\par
    
    \begin{Center}
    Fall 2018
    \end{Center}\par
    
    \begin{Center}
    {\fontsize{14pt}{16.8pt}\selectfont Abstract\par}
    \end{Center}\par
    
    {\fontsize{10pt}{12.0pt}\selectfont Researchers are wanting to know more about how the environment uses solar radiation and how climate around the world is changing. Wireless solar sensors that can be constantly recording data in the field and then uploading the data to a remote data server are seen as a great way to help further solar radiation research. During this project, we will design and implement a 3-D printed solar radiation sensor that can upload accurate data to a remote server. One of our main goals in this project is making the sensor cheap to produce, and 3-D printing the sensor with open-source CAD files will help keep the cost down. Researchers will be able to 3-D print their own solar radiation sensors using our CAD files and deploy them to the field in a cost-effective manner.\par}\par
    
    \newpage
    
    \vspace{\baselineskip}\begin{Center}
    {\fontsize{14pt}{16.8pt}\selectfont Problem Definition\par}
    \end{Center}\par
    
    {\fontsize{10pt}{12.0pt}\selectfont The earth survives because the sun supplies constant energy to countless living organisms. Researchers want to know more about how the environment is using solar radiation and how the earth’s climate is changing. Currently, solar radiation sensors are expensive and produce skewed data due to sensors catching varying concentrations of solar radiation. Researchers need a low-cost method of gathering accurate solar radiation data. In order for researchers to correctly study how the environment is using solar radiation and how the climate is changing, they need to be able to gather data on the three different types of solar radiation; those three types being infrared, visible spectra, and ultraviolet. Researchers are looking to gather large amounts of real-world data, meaning they are looking for data that has been acquired out in the environment, not data acquired in a lab.\par}\par
    
    {\fontsize{10pt}{12.0pt}\selectfont Raw data is typically not useful until it is visualized into something that draws connections and can be processed by humans. The solar radiation data needs to be translated in a way that is useful to the scientists researching how solar radiation is used in relation to the environment and climate change. At the bare minimum, the data needs to be converted to watts per square meter, which is a unit of measurement that can be used to visualize the data. Researchers also need an easy way to access the data, so there needs to be a centralized data hub that the solar radiation data is sent to.\par}\par
    
    \begin{Center}
    {\fontsize{14pt}{16.8pt}\selectfont Proposed Solution\par}
    \end{Center}\par
    
    {\fontsize{10pt}{12.0pt}\selectfont We propose to create a 3-D printed, wireless, open-source radiation sensor. The sensor being 3-D printed will keep it low cost, and having the computer-aided design (CAD) files be open source will help keep the price of maintenance and development down. The sensor will need to be able to detect infrared, visible spectra, and ultraviolet solar radiation. It will also need to be able to interface with an Arduino. Using embedded C, the Arduino microcontroller will handle translating the solar radiation data into watts per square meter and sending the data to a centralized data server. The server is already set up and ready in the Openly Published Environmental Sensing (OPEnS) lab. The sensor will also need to be able to convert the solar radiation to heat. We will use CAD software to design and develop such a sensor. The CAD files will be made open source to allow other engineers and researchers to use and improve the sensor design.\par}\par
    
    {\fontsize{10pt}{12.0pt}\selectfont To solve the issue with solar radiation sensors reporting inaccurate data due to receiving various concentrations of light, we will build a 3-D printed enclosure to go around the sensor. The enclosure will disperse light in an even pattern, thus allowing the solar radiation sensor to receive even amounts of light. We may need to do some server-side programming on the data server so that it knows how to handle the data it receives from the solar radiation sensor. There are several wireless sensors reporting to the data server, so the server needs to know how to handle and properly store the data received by the solar radiation sensor. The reason we are making the sensor wireless is to allow researchers the ability to deploy these sensors in the field. Researchers will not be restricted to keeping these sensors in a lab or at a field station.\par}\par
    
    
    \vspace{\baselineskip}
    \begin{Center}
    {\fontsize{14pt}{16.8pt}\selectfont Performance Metrics\par}
    \end{Center}\par
    
    {\fontsize{10pt}{12.0pt}\selectfont This is a project with a clear objective and it will be clear when the project is considered complete. The first thing the team needs to produce is a proven and tested prototype of the solar radiation sensor. A completed sensor will be able to detect and record the watts per square meter data on infrared, ultraviolet, and visible spectra solar radiation. It will also be able to communicate and send data to a data server as well as convert the solar radiation to heat. The sensor needs to be 3-D printed and the CAD files need to be properly documented and made open source. The second thing the team needs to produce is a 3-D printed enclosure that helps ensure the solar radiation data is accurate. By the end of the project, the team will have a solar radiation sensor that uploads accurate data to the data server.\par}\par
    
    \end{document}