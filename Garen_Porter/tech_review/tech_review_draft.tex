\documentclass[10pt,draftclsnofoot,onecolumn,letterpaper]{article}
    \usepackage{ragged2e}
    \usepackage[svgnames,table]{xcolor}
    \usepackage[hidelinks]{hyperref}
    
    %\usepackage{pgfgantt}
    \usepackage{graphicx}
        
    \usepackage[utf8]{inputenc}
    \usepackage[left=0.75in, right=0.75in, top=0.75in]{geometry}
        
    \usepackage[T1]{fontenc}
    \setlength{\parindent}{0pt}
        
        
    \begin{document}
    \begin{Center}
    {\fontsize{14pt}{16.8pt}\selectfont OPEnS Pyranometer: IoT Solar Radiation Sensor\\ Tech Review\par}
    \end{Center}\par
        
    \begin{Center}
    Software Engineering Project: Fall 2018\\Garen Porter\\
    \end{Center}\par
    \hrule
    \begin{Center}
    {\fontsize{12pt}{16.8pt}\selectfont \textbf{Abstract}\par}
    \end{Center}\par
        
    {\fontsize{10pt}{12.0pt}\selectfont Three different parts and systems that will need to be considered in our pyranometer are the power source, the optical black object, and the data transmission system. There are a few different options and approaches for each of the three pieces, each with pros and cons. For data transmission, LoRa radio, Wifi, and cellular networks will be explored. Each form of data transmission has trade-offs between bandwidth, range, and power consumption. For the optical black object, the properties of Vantablack, polytetrafluoroethylene (Teflon), and Polyethylene Terephthalate (PET) are analyzed and compared to find an optimal material. Finally, lithium batteries, solar panels, and thermoelectric generators are each looked at to decide if they are viable power sources for the pyranometer. 
    \par}\par
        
    \newpage
        
        
    {\fontsize{12pt}{12.0pt} \textbf{Introduction}\\\selectfont 
    \par}\par
    {\fontsize{12pt}{12.0pt} There several different components that will be going into our open-source pyranometer, and not surprisingly there are multiple implementations for each component. More specifically, the data transmission system, the optical black object, and the power source each have a few different implementations. Each implementation has pros and cons and may or may not be suitable for the open-source pyranometer. After the evaluation of each implementation, a conclusion will be met about which implementations will suit the project best.\\\selectfont 
    \par}\par
    
    {\fontsize{10pt}{12.0pt} \textbf{Technology 1: Data Transmission}\\Each implementation of data transmission will have trade-offs between bandwidth, range, and power consumption [1]. For the open-source pyranometer, an implementation with low power consumption and high range is preferred, but three different options, each with its own strengths, will be looked at.\\\selectfont 
    \par}\par
    {\fontsize{10pt}{12.0pt}\textit{LoRa}\\\selectfont 
    \par}\par
    {\fontsize{10pt}{12.0pt}LoRa is a wireless radio communications technology. In general, it has a long range, low power consumption, and low bandwidth [1]. This means it transmits data slowly, but it does so without consuming much power and over long distances (between 10 and 20 kilometers). LoRa uses what is called chirp spread spectrum to transmit data [2]. Chirp spread spectrum sends multiple waves along the same radio frenquency, and uses changes in wave frequency to represent binary data [2]. Sending data in this way reduces power consumption and helps prevent interference. A LoRa radio chip costs about \$35 and can be easily integrated into a microcontroller. Using LoRa will allow users to deploy the pyranometer in remote locations for longer periods of time at the cost of receiving data slowly.\\\selectfont 
    \par}\par
    {\fontsize{10pt}{12.0pt}\textit{Wifi}\\\selectfont 
    \par}\par
    {\fontsize{10pt}{12.0pt}Wifi, in general, has a short range, has low power consumption, and has high bandwidth [1]. This means wifi is able to send data quickly over a short distance while consuming low amounts of power. Wifi's range is only about 1000 meters [3], but it can send a lot of data in a short amount of time. Wifi is popular in IoT devices that operate within homes, such as smart appliances and home monitoring devices [3]. This is due to smart home devices being in close proximity to a router and to each other, so the limited range of Wifi is not as much of a concern. The pyranometer that is being developed will be deployed in wilderness areas and will often need to communicate over a distance of over 10 kilometers. High bandwidth is also not necessary, as readings are only needed every few seconds, so constant data transmission is not necessary. For these reasons, Wifi is most likely not the best option for data transmission.\\\selectfont 
    \par}\par
    {\fontsize{10pt}{12.0pt}\textit{Cellular Networks}\\\selectfont 
    \par}\par
    {\fontsize{10pt}{12.0pt}Cellular networks offer high range, high bandwidth, and high power consumption [1]. A major drawback to using cellular networks is that the pyranometer will need to be in range of a cell tower, which limits the areas in which the pyranometer can be deployed. If the pyranometer is deployed within range of a tower, then it will be able upload large amounts of data to the internet quickly, but it will consume a lot of power in the process. The pyranometer is not meant to upload large amounts of data, but rather upload small samples of data every few seconds. Using cellular networks as the form of data transmission would greatly limit the areas that the pyranometer could be deployed, and it would limit the device's life in the field due to high power consumption, so cellular networks are not a good option for data transmission.\\\selectfont 
    \par}\par
    
    {\fontsize{10pt}{12.0pt} \textbf{Technology 2: Optical Black}\\The thermopile that the pyranometer uses to report solar radiation levels needs to read solar radiation emissions from another object. The best way to implement this is by using a black object with high emissivity (an object's ability to emit thermal radiation) [4]. Three different materials will be examined for their emissivity, price, and their ability to be shaped and formed.\\\selectfont 
    \par}\par
    \newpage
    {\fontsize{10pt}{12.0pt}\textit{Vantablack and Black 2.0}\\\selectfont 
    \par}\par
    {\fontsize{10pt}{12.0pt}Vantablack is the blackest material ever created [5]; it has an emissivity value of 1.0 and it absorbs 99.96\% of all light [4]. Vantablack is made up of billions of evenly spaces carbon nanotubes. When light hits the Vantablack, it gets trapped bouncing between the nanotubes and is converted to heat [5]. Ideally, Vantablack would be used as the optical black for the thermopile because it would give the most accurate readings possible, but it is impossible to acquire without a special license. The next best thing is Black 2.0. Black 2.0 was inspired by Vantablack, but rather than being a physical object, it is a paint that can be applied to most surfaces [6]. It is reportedly the blackest material that can be bought commercially. It is about \$20 per tube and can theoretically be applied to any object and be used as an optical black. It will be possible to 3D print an object of any desired shape, coat it with Black 2.0, and use it as an optical black object for use by the thermopile. With Black 2.0 being only \$20 and being able to use it on any object makes it a viable option for use as an optical black object.\\\selectfont 
    \par}\par
    {\fontsize{10pt}{12.0pt}\textit{Polytetrafluoroethylene}\\\selectfont 
    \par}\par
    {\fontsize{10pt}{12.0pt}Polytetrafluoroethylene, better known as Teflon, is a plastic fluoropolymer commonly used to coat cook-ware and contain corrosive chemicals [7]. Teflon is able to be cut to any desired size using a laser cutter, is chemically inert, and has a high melting point of 327 degrees celcius [7]. All in all, Teflon is strong, durable, and able to shaped to any desired size (with use of a laser cutter). Teflon has an emissivity coefficient of 0.92 [4], which is fairly high. Of course, the Teflon would need to be black in order to be most effective as an optical black object for the thermopile. It is about \$20 per sheet (12"x24"x0.03"), making it rather expensive when you factor in having to laser cut it. Overall, Teflon is a viable option due its durability and emissivity, but it is moderately expensive and difficult to cut.\\\selectfont 
    \par}\par
    {\fontsize{10pt}{12.0pt}\textit{Polyethylene Terephthalate}\\\selectfont 
    \par}\par
    {\fontsize{10pt}{12.0pt}Polyethylene Terephthalate, better known as PET, is one of the most used plastic materials in the world and is most popular for its use in disposable water bottles [8]. PET is a thermoplastic with a relatively low melting point of 260 degrees celcius [8]. Due to PET's low melting point, it is a popular material used in 3D printing. Ideally, black PET could be 3D printed into any desired shape and be used as an optical black object for the thermopile. Unfortunately, with an emissivity value of 0.80 [4], PET has low emissivity compared to other plastics. A spool of PET is around \$16, which is inexpensive compared to other options. PET gives the freedom of easily shaping it into whatever shape is needed, and it has a low price, but its emissivity value is too low to be considered as an optical black.\\\selectfont 
    \par}\par
    
    {\fontsize{10pt}{12.0pt} \textbf{Technology 3: Power Source}\\When choosing a power source, the main characteristics that are looked at are longevity and price. It is important to keep in mind that taking the time to replace or repair power sources costs money, so power sources that need to be replaced often, while perhaps not expensive to purchase, are expensive due to having to replace them constantly. The feather M0 microcontroller takes 3.3V, so any power source must produce at least 3.3V. For the open-source pyranometer, a power source that lasts a long time relative to its price is most ideal.\\\selectfont 
    \par}\par
    {\fontsize{10pt}{12.0pt}\textit{Solar Panel}\\\selectfont 
    \par}\par
    {\fontsize{10pt}{12.0pt} Solar panels are generally a good way to keep a battery charged and require little human interaction. The goal with a solar panel is to hook it up and leave it be. Ideally, users would set it up in the field and leave it for years at a time. Voltaic creates solar panels that are made specifically for IoT devices [9]. Their panels are made of aluminum-plastic composite substrate, which makes them light and durable [9]. They reportedly last 8-10 years and are completely weatherproof [9]. They have several different models, but the model that would best suit the pyranometer is the 1-watt solar panel. It is small (3.5"x4.4"x0.2"), outputs 6.5V, and has an efficiency of 19\% [9]. The panel costs \$19, which is not too expensive, but it is required to be hooked up to a battery that has a solar charge circuit [9]. Voltaic sells such batteries and they cost about \$30. All together the package would cost \$49. Voltaic's battery-solar panel combination has features in place that benefit IoT devices. When the battery dies, it will not power the device on until the battery builds up a reasonable amount of charge [9]. This prevents the battery from flickering the device on and off. The battery, unless dead, always outputs voltage regardless of whether or not the IoT device is drawing power [9]. The solar panel is relatively affordable for how long it lasts, but it is only effective in a sunny climate. The open-source pyranometer is meant to be general purpose and cheap, so a solar panel with a solar charged battery are most not likely not the best power solution. However, the solar panel solution will be desired by some users, so documentation on how to integrate a solar panel can be included in the pyranometer documentation. \\\selectfont 
    \par}\par
    {\fontsize{10pt}{12.0pt}\textit{Lithium Battery}\\\selectfont 
    \par}\par
    {\fontsize{10pt}{12.0pt}Lithium is a light metal and has the greatest energy density (power to weight ratio) of all chemical batteries [10]. This makes it the most effective chemical battery. Unlike alkaline batteries, chemical batteries convert chemical energy to electrical energy [11]. The typical "coin cell" lithium battery costs about \$4.00 and lasts about 50 hours (with constant use) before needing to be recharged [11]. This means that the battery needs to be changed or charged every 2 days. Despite needing to be changed out often, the batteries are cost effective. One lithium battery can last several years depending on its discharge rate [10]. Each time a lithium battery is recharged it loses some capacity due to self-discharge [10]. Going with lithium batteries as the power source would be easy and cheap to implement, but it will require consistent recharging. The cost effectiveness of Lithium batteries may make it ideal for the open-source pyranometer.\\\selectfont 
    \par}\par
    {\fontsize{10pt}{12.0pt}\textit{Thermoelectric Generator}\\\selectfont 
    \par}\par
    {\fontsize{10pt}{12.0pt}Thermoelectric generators are devices that turn heat into electricity. Thermoelectric generators are able to convert changes in temperature into electricity [12]. If one end of the generator is hot and the other end is cold, electrons will flow from the hot end to the cold end and generate electricity [12]. The amount of electricity generated is relative to the temperature difference, and the efficiency of energy transfer is limited, so thermoelectric generators are only effective when there is a significant difference in temperature between both ends of the generator [12]. A small thermoelectric generator suitable for IoT devices are about \$45 dollars, which is a rather expensive purchase, but should last a long time [13]. A small thermoelectric generator cannot power a microcontroller by itself, but it could be used to increase the life of a rechargeable lithium battery [13]. The generator could be hooked up to a lithium battery and charge it continually using the heat radiated from the optical black object. If the thermoelectric generator is able to increase the life of a lithium battery by a reasonable amount of time, it may be well worth the purchase.\\
    \par}\par
    
    {\fontsize{12pt}{12.0pt} \textbf{Conclusion}\\\selectfont 
    \par}\par
    {\fontsize{12pt}{12.0pt} Three pieces of the open-source pyranometer were explored, and three implementations of each piece were analyzed to determine what will work best. For data transmission, LoRa, Wifi, and cellular networks were looked at. LoRa was determined to be the best technology to use due to its low power consumption and high range. For the optical black object, Black 2.0, Polytetrafluoroethylene (Teflon), and Polyethylene Terephthalate (PET) were looked at. It is determined that Teflon and Black 2.0 (coated over a 3D printed object) would both be good options, but Black 2.0 is a slightly better option due to its ease of use and flexibility in shape. Powering the pyranometer is the trickiest of the three pieces to analyze, as there is no clear best power source to use. From the research done, it looks like a lithium battery combined with a device to charge it is the best option. This will most likely be a thermoelectric generator constantly using the heat from the optical black to charge the lithium battery, thus increasing its life in the field.\\\selectfont 
    \par}\par
    
    \newpage
    
    {\fontsize{12pt}{12.0pt} \textbf{References}\\\selectfont 
    \par}\par
    
    {\fontsize{12pt}{12.0pt}
    [1] C. McClelland, “IoT Connectivity 101,” \textit{IoT For All}, 13-Feb-2018. [Online]. Available:\\
    https://internetofthingsagenda.techtarget.com/feature/IoT-battery-outlook-Types-of-batteries-for-IoT-devices. [Accessed: 03-Nov-2018].\\
    
    [2] MickMake, "\#145 What is LoRa?," \textit{Tutorial QuickBits}, 06-18-2017. [Online]. Available:\\ https://www.youtube.com/watch?reload=9\&v=-d2JxZuvTOI. [Accessed: 03-Nov-2018].\\
    
    [3] N. Raman, “How low-powered Wi-Fi sensors are the future of the IoT,” \textit{Imagination}, 10-Jul-2017. [Online]. Available: https://www.imgtec.com/blog/how-low-powered-wi-fi-sensors-are-the-future-of-iot/. [Accessed: 03-Nov-2018].\\
    
    [4] na, “Emissivity Coefficients Materials,” \textit{Engineering ToolBox}, 2003. [Online]. Available:\\ https://www.engineeringtoolbox.com/emissivity-coefficients-d\_447.html. [Accessed: 03-Nov-2018].\\
    
    [5] T. Doran and K. Scott, “‘Darkest building on Earth’ unveiled at Winter Olympics,” \textit{CNN}, 12-Feb-2018. [Online]. Available: https://www.cnn.com/2017/11/15/world/vantablack-blackest-black-material/index.html. [Accessed: 03-Nov-2018].\\
    
    [6] S. Cascone, “Vantablack vs. Black 2.0: Which Is the Superblack for You?,” \textit{artnet News}, 30-Mar-2017. [Online]. Available: https://news.artnet.com/art-world/vantablack-vs-black-superblack-907556. [Accessed: 03-Nov-2018].\\
    
    [7] na, “The Properties and Advantages of PTFE,” \textit{AFT Fluorotec}, 19-Jul-2016. [Online]. Available:\\ https://www.fluorotec.com/blog/the-properties-and-advantages-of-polytetrafluoroethylene-ptfe/. [Accessed: 03-Nov-2018].\\
    
    [8] na, “Everything you Need to Know About The World’s Most Useful Plastic (PET and Polyester),” \textit{Creative Mechanisms}, 20-Jun-2016. [Online]. Available: https://www.creativemechanisms.com/blog/everything-about-polyethylene-terephthalate-pet-polyester. [Accessed: 03-Nov-2018].\\
    
    [9] na, “IoT and Remote Sensors,” \textit{Voltaic Systems}, nd. [Online]. Available: https://www.voltaicsystems.com/iot. [Accessed: 03-Nov-2018].\\
    
    [10] B. Schweber, “Options for Powering Your Wireless IoT Device” \textit{Digi-Key}, 22-Apr-2016. [Online]. Available: https://www.digikey.com/en/articles/techzone/2016/apr/options-for-powering-your-wireless-iot-device. [Accessed: 03-Nov-2018].\\
    
    [11] W. Rowe, “IoT battery outlook: Types of batteries for IoT devices,” \textit{IoT Agenda}, nd. [Online]. Available: https://internetofthingsagenda.techtarget.com/feature/IoT-battery-outlook-Types-of-batteries-for-IoT-devices. [Accessed: 03-Nov-2018].\\
    
    [12] J. Chu, “Turning heat into electricity,” \textit{MIT News}, 16-Jan-2018. [Online]. Available:\\ http://news.mit.edu/2018/topological-materials-turning-heat-electricity-0117. [Accessed: 03-Nov-2018].\\
    
    [13] na, “EverGenTM Mini-Harvesters - Marlow” \textit{Digi-Key}, nd. [Online]. Available:\\ https://www.digikey.com/en/product-highlight/m/marlow/evergen-mini-harvesters. [Accessed: 03-Nov-2018].\\
    
    \selectfont 
    \par}\par
    
    \end{document}